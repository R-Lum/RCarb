\begin{table}[ht]
\centering
\begin{tabular}{rllllllll}
  \hline
 & Name & Title & Description & Version & m.Date & m.Time & Author & Citation \\ 
  \hline
1 & Example\_Data & Example data & Example data as shipped with  Carb  by Mauz \& Hoffmann (2014). In contrast to the original data,  NA  values have been replaced by 0 and columns and rows have been transposed. Samples are now organised in rows and parameters in columns.  The data can be used to test  'RCarb'  and play with the secondary carbonatisation process. Sample HD107 was renamed to LV107 for the sake of consistency with Fig. 4 in Mauz \& Hoffmann (2014). &  &  &  & Mauz \& Hoffmann (2014), with minor modifications by Sebastian Kreutzer,Geography \& Earth$<$br /$>$ Sciences, Aberystwyth University (United Kingdom)$<$br /$>$ &  \\ 
  2 & model\_DoseRate & Model dose rate evolution in carbonate-rich samples & This function models the dose rate evolution in carbonate enrich environments. For the calculation internal functions are called. & 0.2.1
 &  &  & Sebastian Kreutzer, Institute of Geography, Heidelberg University (Germany); based$<$br /$>$ on 'MATLAB' code given in file Carb\_2007a.m of  Carb $<$br /$>$ &  \\ 
  3 & Reference\_Data & Reference data & Reference data and correction factors for beta and gamma radiation used for internal calculations. These values are used instead of the correction factors given in Aitken (1985) for the carbonate model. &  &  &  &  &  \\ 
  4 & write\_InputTemplate & Write table input template & This function creates a template table that can be used as input for the function model\_DoseRate &  & 0.1.0
 &  & Sebastian Kreutzer, Institute of Geography, Heidelberg University (Germany)$<$br /$>$ &  \\ 
   \hline
\end{tabular}
\end{table}

